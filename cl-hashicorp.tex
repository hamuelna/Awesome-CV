%!TEX TS-program = xelatex
%!TEX encoding = UTF-8 Unicode
% Awesome CV LaTeX Template for Cover Letter
%
% This template has been downloaded from:
% https://github.com/posquit0/Awesome-CV
%
% Authors:
% Claud D. Park <posquit0.bj@gmail.com>
% Lars Richter <mail@ayeks.de>
%
% Template license:
% CC BY-SA 4.0 (https://creativecommons.org/licenses/by-sa/4.0/)
%


%-------------------------------------------------------------------------------
% CONFIGURATIONS
%-------------------------------------------------------------------------------
% A4 paper size by default, use 'letterpaper' for US letter
\documentclass[11pt, a4paper]{awesome-cv}

% Configure page margins with geometry
\geometry{left=1.4cm, top=.8cm, right=1.4cm, bottom=1.8cm, footskip=.5cm}

% Specify the location of the included fonts
\fontdir[fonts/]

% Color for highlights
% Awesome Colors: awesome-emerald, awesome-skyblue, awesome-red, awesome-pink, awesome-orange
%                 awesome-nephritis, awesome-concrete, awesome-darknight
\colorlet{awesome}{awesome-red}
% Uncomment if you would like to specify your own color
% \definecolor{awesome}{HTML}{CA63A8}

% Colors for text
% Uncomment if you would like to specify your own color
% \definecolor{darktext}{HTML}{414141}
% \definecolor{text}{HTML}{333333}
% \definecolor{graytext}{HTML}{5D5D5D}
% \definecolor{lighttext}{HTML}{999999}

% Set false if you don't want to highlight section with awesome color
\setbool{acvSectionColorHighlight}{true}

% If you would like to change the social information separator from a pipe (|) to something else
\renewcommand{\acvHeaderSocialSep}{\quad\textbar\quad}


%-------------------------------------------------------------------------------
%	PERSONAL INFORMATION
%	Comment any of the lines below if they are not required
%-------------------------------------------------------------------------------
% Available options: circle|rectangle,edge/noedge,left/right
\photo[circle,noedge,left]{./ham_profile.jpeg}
\name{Worapol}{Boontanonda}
\position{MSc Student{\enskip\cdotp\enskip} Ex Data Engineer}
\address{Flat 28 Canal point, 22 West Tollcross Street, Edinburgh, United Kingdom}

\mobile{(+44)7387507456}
\email{worapol.study@gmail.com}
\homepage{www.hamuel.me}
\github{hamuelna}
\linkedin{worapol-boontanonda}
\stackoverflow{6696549}{hamuel}
% \gitlab{gitlab-id}
% \stackoverflow{SO-id}{SO-name}
% \twitter{@twit}
% \skype{skype-id}
% \reddit{reddit-id}
% \extrainfo{extra informations}

% \quote{``Be the change that you want to see in the world."}


%-------------------------------------------------------------------------------
%	LETTER INFORMATION
%	All of the below lines must be filled out
%-------------------------------------------------------------------------------
% The company being applied to
\recipient
  {Company Recruitment Team}
  {HashiCorp\\United Kingdom}
% The date on the letter, default is the date of compilation
\letterdate{\today}
% The title of the letter
\lettertitle{Job Application for Backend Engineer - Cloud Services}
% How the letter is opened
\letteropening{Dear Recruitment Team}
% How the letter is closed
\letterclosing{Sincerely,}
% Any enclosures with the letter
% \letterenclosure[Attached]{Curriculum Vitae}


%-------------------------------------------------------------------------------
\begin{document}

% Print the header with above personal informations
% Give optional argument to change alignment(C: center, L: left, R: right)
\makecvheader[R]

% Print the footer with 3 arguments(<left>, <center>, <right>)
% Leave any of these blank if they are not needed
\makecvfooter
  {\today}
  {Worapol Boontanonda~~~·~~~Cover Letter}
  {}

% Print the title with above letter informations
\makelettertitle

%-------------------------------------------------------------------------------
%	LETTER CONTENT
%-------------------------------------------------------------------------------
\begin{cvletter}

\lettersection{About Me}
My name is Worapol Boontanonda and currently, I am pursuing an MSc in High Performance Computing.
I will finish my dissertation at the end of August 2022 and should be able to start working from
September 2022 onwards.
Before studying for the MSc I have worked as a Data Engineer for 3 years at Toyota Tsusho Nexty Electronics 
however, I have performed a lot of tasks that are unrelated to my role such as Fullstack Development 
and DevOps on the AWS Cloud. I believed my past experiences have proven me to be a person 
with a can-do attitude. Although I need to perform these unrelated tasks I still manage to do them successfully. 
Therefore, with this mindset, I can solve any engineering/software problem with the right tools. 
Unfortunately, having the correct mindset is not enough 
because having the correct skillset and expertise is also important. 
I hope that this MSc will help fill the missing piece in my skillset to progress my career further. I am currently studying how to optimise and write maintainable software.

\lettersection{Why HashiCorp?}

HashiCorp is a pioneer in different tools and technologies that make development in the cloud easier and more maintainable. One of the biggest product from HashiCorp is Terraform one of the first infrastructure as a code (IaaC) framework used in many production systems around the world. I have also used Terraform to deploy a Kubernetes cluster to AWS EKS service. Terraform is more flexible than other solution because it can support deployment on many cloud services. I hope to be part of a team that will take the next step on making these technologies easier to use and requires less maintenance from the user.

\lettersection{Why Me?}
I think I have enough skill and experience in both HashiCorp products such as Terraform and using the cloud. This is because I have at least 3 years of experience using AWS to develop application on the cloud environment, and I was exposed to lots of different cloud services. Furthermore, I have 2 AWS associate certification in order to ensure that I apply that best practices when I develop applications on the cloud. I have written some Golang application to solve a vehicle route optimization problem and expose it as a REST API service via AWS API Gateway and AWS Lambda. I am hoping to apply some of my pain point when I was using Terraform to create a better product for HashiCorp.

\end{cvletter}


%-------------------------------------------------------------------------------
% Print the signature and enclosures with above letter informations
\makeletterclosing

\end{document}
